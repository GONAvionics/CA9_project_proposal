\documentclass[10pt,a4paper,oneside,openany,article]{memoir}

\usepackage[english]{babel}
\usepackage[utf8]{inputenc}
\usepackage[T1]{fontenc}

\setlength{\parindent}{1.5mm}
\setlength{\parskip}{1.8mm}

\title{\vspace{-2cm} Cubesat Attitude Control with Supervisory Control System \\ \Large - Project Proposal}
\author{Brian Gasberg Thomsen \& Jens Nielsen}

\begin{document}
\maketitle

\chapter{Introduction}
The development in the CubeSat industry demands increasing density of different technologies in a single unit CubeSat. This means elevated power consumption and maybe even higher demands for potential direction sensitive equipment like a camera, patch antenna or thrusters. 

Traditional methods of attitude control in CubeSats uses actuators like magnetorquers, reaction wheels and/or thrusters for stability and active pointing. Each kind of actuator has its own advantages and disadvantage when it comes to 

\chapter{Problem Description}
The project can include:
\begin{itemize}\tightlist
  \item Nonlinear control of a cubesat
  \item Supervisory control system design
  \item Prototype design
  \item Test stand
  \item ...
\end{itemize}

\begin{description}
  \item[Nonlinear Control:] A nonlinear control strategy is designet and verified through simulations and hopefully in the prototype.
  \item[Supervisory Control System:] A supervisory control strategy is proposed; when a cubesat satellite is operated, e.g. with several attitude sensitive payloads it might be beneficial to look into a method og optimising after e.g. the power budget which is often limited in a cubesat mission.
  \item[Prototype Design:] 
  \item[Test stand:] 
\end{description}



\end{document}
