\documentclass[10pt,a4paper,oneside,openany,article]{memoir}

\usepackage[english]{babel}
\usepackage[utf8]{inputenc}
\usepackage[T1]{fontenc}

\setlength{\parindent}{1.5mm}
\setlength{\parskip}{1.8mm}

\title{\vspace{-2cm} Cubesat Attitude Control with Supervisory Control System \\ \Large - Project Proposal}
\author{Brian Gasberg Thomsen \& Jens Nielsen}

\begin{document}
\maketitle

\chapter{Introduction}
The development in the CubeSat industry demands increasing density of different technologies in a single unit CubeSat. This means elevated power consumption and maybe even higher demands for potential attitude sensitive equipment like a camera, patch antennas or thrusters. 

Traditional methods of attitude control in CubeSats uses actuators like Magnetic Torquers, reaction wheels and/or thrusters for stability and active pointing. Each type of actuator has its own advantages and disadvantages as the level of power consumption relative to moment generation, density and complexity, each which makes them suited for different missions with a variety of control scenarios. The specifications of the actuators is also under the restrictions of the CubeSats volume and energy capacity which requires small and conservative actuators.

The Attitude Control System (ACS) for the CubeSat standard with the mentioned constrains, is vital to other systems in the CubeSat both is managing the attitude of the spacecraft but also constrained by a power budget, e.g. when the ACS is done actuating there has to be enough energy capacity to fulfil other objectives like transmitting data or using the payload.

Such an ACS should be aware of the spacecraft's objectives to come and their importances in order to optimise the required control manoeuvres and thereby the power consumption relative to the energy budget.

\chapter{Problem Description}
The project can include:
\begin{itemize}\tightlist
  \item Nonlinear control of a cubesat
  \item Supervisory control system design
  \item Prototype design
  \item Test stand
  \item ...
\end{itemize}

\begin{description}
  \item[Nonlinear Control:] A nonlinear control strategy is designet and verified through simulations and hopefully in a prototype. Depending on the mission, different actuators are relevant, this includes:
    \begin{itemize}\tightlist
      \item Magnetic torquers
      \item Reaction wheels
      \item Pulsed plasma thruster (PPT)
    \end{itemize}
    A combination of magnetic torquers and reaction wheels can be considered for attitude control and if the mission requires an Orbit Control Systems (OCS) a PPT might be implemented, this however greatly influences the ACS af the satellite and can be investigated.
  \item[Supervisory Control System:] A supervisory control strategy is proposed; when a cubesat satellite is operated, e.g. with several attitude sensitive payloads it might be beneficial to look into a method of optimising after e.g. the power budget which is often limited in a cubesat mission. This can make the spacecraft (SC) more autonomous because this makes it posible for the SC to perform the actions in its flightplanner taking the power constraints in consideration.
  \item[Prototype Design:] An early prototype of the ACS board can be manufactured in order to verify the control algorithms.
  \item[Test stand:] In order to verify the control algorithms a test stand needs to be manufactured.
\end{description}



\end{document}
